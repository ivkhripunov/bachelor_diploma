\section{Введение}
\label{sec:Chapter0} \index{Chapter0}

\subsection*{Актуальность работы}

На сегодняшний день количество объектов на околоземных орбитах стремительно увеличивается.
Для мониторинга техногенных космических объектов необходимо
определять орбитальные параметры космических аппаратов и космического мусора,
прогнозировать траектории и предсказывать опасные сближения.
Таким образом, возникает необходимость создания программного 
комплекса для уточнения орбит космических аппаратов. 
Ключевыми требованиями к разработке программного 
комплекса являются повышение точности, 
сокращение вычислительных затрат и увеличение быстродействия.

\subsection*{Цель работы}

Целью настоящей работы является создание точного и высокопроизводительного 
программного комплекса для уточнения орбит космических аппаратов.

\subsection*{Задачи}

\begin{enumerate}
    \item Анализ существующих физико-математических моделей движения тел 
    в околоземном космическом пространстве, определение наиболее ресурсоемких моделей.
    \item Создание методов и алгоритмов ускорения прогноза околоземных орбит 
    за счет увеличения быстродействия вычисления плотности атмосферы Земли.
    \item Реализация алгоритмов прецизионного восстановления орбиты с использованием
    псевдоускорений и пседовимпульсов.
    \item Валидация программного комплекса.
\end{enumerate}

\subsection*{Научная новизна}

В настоящей работе впервые предложено использование четырехмерной интерполяции для
ускорения расчета плотности верхней атмосферы Земли 
при интегрировании уравнений движения тел в околоземном космическом пространстве.

\subsection*{Практическая значимость}

Работа обладает высокой практической значимостью. 
Применение предлагаемых методов позволяет существенно сократить ресурсоемкость алгоритмов 
прогноза движения космических объектов, используемых при решении 
многих практических задач освоения космического пространства, в особенности в задаче 
мониторинга техногенных космических объектов. Предлагаемые методы позволяют до 6 раз 
повысить быстродействие алгоритмов прогнозирования околоземных орбит.

Высокоточные эфемериды космического аппарата, полученные при восстановлении орбиты,
могут быть использованы для калибровки наземных измерительных средств и
уточнения параметров вращения Земли. 
Кроме того, модель движения с псевдоимпульсами и псевдоускорениями применима 
для определения маневров космических аппаратов.

% \subsection*{Личный вклад}

\newpage
