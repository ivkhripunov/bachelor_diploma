\section{Введение}

\label{sec:Chapter0} \index{Chapter0}

\subsection*{Актуальность работы}

На сегодняшний день количество объектов на околоземных орбитах стремительно увеличивается.
Для мониторинга техногенных космических объектов необходимо
определять орбитальные параметры космических аппаратов и космического мусора,
прогнозировать траектории и предсказывать опасные сближения.
Рост числа космических объектов приводит к перегрузке существующих систем 
контроля околоземного пространства.
Таким образом, возникает необходимость создания высокопроизводительного программного 
комплекса для уточнения орбит космических аппаратов. 
Ключевыми требованиями к разработке программного 
комплекса являются повышение точности, 
сокращение вычислительных затрат и увеличение быстродействия.

\subsection*{Цель работы}

Целью настоящей работы является создание высокопроизводительного 
программного комплекса для восстановления орбит и уточнения орбитальных параметров 
космических аппаратов.

\subsection*{Задачи}

\begin{enumerate}
    \item Анализ существующих физико--математических моделей движения тел 
    в околоземном космическом пространстве, определение наиболее ресурсоемких моделей.
    \item Ускорение баллистического прогноза за счет создания методов и алгоритмов 
    увеличения быстродействия вычисления плотности атмосферы Земли.
    \item Дополнение существующих алгоритмов высокоточного восстановления орбиты моделью 
    движения c псевдоускорениями и псевдоимпульсами.
    \item Валидация программного комплекса.
\end{enumerate}

\subsection*{Научная новизна}

В настоящей работе впервые предложено использование четырехмерной интерполяции для
ускорения расчета плотности верхней атмосферы Земли 
при интегрировании уравнений движения тел в околоземном космическом пространстве.

\subsection*{Практическая значимость}

Применение предлагаемых методов увеличения быстродействия вычисления плотности атмосферы 
Земли позволяет существенно сократить ресурсоемкость алгоритмов 
прогноза движения космических объектов, используемых при решении 
многих практических задач освоения космического пространства, в особенности, в задаче 
мониторинга техногенных космических объектов. Предлагаемые методы позволяют кратно
повысить быстродействие алгоритмов прогнозирования околоземных орбит.

Дополнение существующих алгоритмов высокоточного восстановления орбиты моделью 
движения c псевдоускорениями и псевдоимпульсами позволило провести валидацию
программного комплекса и сравнить ошибки позиционирования на мерном интервале при
использовании различных моделей сил. В перспективе методы могут быть применены для детекции маневров и
получения высокоточных эфемерид глобальных навигационных спутниковых систем.

\subsection*{Личный вклад}

В ходе выполнения работы автор:
\begin{enumerate}
    % \item Аналитический алгоритм, позволяющий определять границы временных 
    % промежутков нахождения космических объектов в тени и полутени.
    \item Реализовал модели верхней атмосферы ГОСТ 25645.166–2004, \\
    NRLMSISE-00, NRLMSIS-21
    с возможностью использования актуальных данных космической погоды.
    \item Адаптировал существующий алгоритм многомерной интерполяции для
    аппроксимации моделей атмосферы.
    \item Разработал метод подбора рациональных параметров интерполянта,
    обеспечивающих приемлемую точность прогноза при минимальных затратах памяти.
    \item Дополнил существующие алгоритмы высокоточного восстановления орбиты мо-
    делью движения c псевдоускорениями и псевдоимпульсами.
    \item Валидировал дополненный программный комплекс на орбите космического аппарата LAGEOS-2.
\end{enumerate}


\newpage
