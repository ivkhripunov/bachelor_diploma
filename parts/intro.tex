\section{Введение}
\label{sec:Chapter0} \index{Chapter0}

\subsection*{Актуальность работы}

На сегодняшний день количество объектов на околоземных орбитах стремительно увеличивается. 
По этой причине возникает потребность в мониторинге техногенных космических объектов – 
определении орбитальных параметров космических аппаратов и космического мусора, 
прогнозировании их траекторий и предупреждении опасных ситуаций.
Для удовлетворения перечисленных нужд необходима разработка математических методов, 
алгоритмов и программных комплексов прогнозирования движения тел в околоземном космическом 
пространстве. Одними из главных требований, увеличение их точностных характеристик, 
сокращение потребных вычислительных затрат и повышение быстродействия.

\subsection*{Цель работы}

Целью настоящей работы является создание программного комплекса для уточнения орбит 
космических аппаратов.

\subsection*{Задачи}

\begin{enumerate}
    \item Анализ существующих физико-математических моделей движения тел 
    в околоземном космическом пространстве, определение наиболее ресурсоемких моделей.
    \item Создание методов и алгоритмов ускорения расчета плотности атмосферы Земли
    \item Реализация алгоритмов прецизионного восстановления орбиты с использованием
    псевдоускорений и пседовимпульсов
    \item Валидация программного комплекса
\end{enumerate}

\subsection*{Научная новизна}

В настоящей работе впервые предложено использование четырехмерной интерполяции для
ускорения расчета плотности верхней атмосферы Земли 
при интегрировании уравнений движения тел в околоземном космическом пространстве.

\subsection*{Практическая значимость}

Работа обладает высокой практической значимостью. 
Применение предлагаемых методов позволяет существенно сократить ресурсоемкость алгоритмов 
прогноза движения тел в околоземном космическом пространстве, используемых при решении 
многих практических задач освоения космического пространства, в особенности в задаче 
мониторинга техногенных космических объектов. Предлагаемые методы позволяют до 6 раз 
повысить быстродействие алгоритмов интегрирования движения космических обектов.

Высокоточные эфемериды космического аппарата, полученные при восстановлении орбиты,
могут быть использованы для калибровки наземных измерительных средств. 
Кроме того, модель движения с псевдоимпульсами и псевдоускорениями применима 
для определения маневров космических аппаратов.

\newpage
