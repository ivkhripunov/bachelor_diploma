\paragraph{Список обозначений и сокращений}

\begin{itemize}
    \item КО -- космический объект
    \item КА -- космический аппарат
    \item СК -- система координат
    \item MEE (Modified Equinotical Elements) -- модифицированные равноденственные элементы
    \item ОДУ -- обыкновенное дифференциальное уравнение
    \item ГНСС -- глобальная навигационная спутниковая система
    \item SLR (Satellite Laser Ranging) -- спутниковая лазерная дальнометрия
    \item DORIS (Doppler Orbitography and Radiopositioning Integrated by Satellite) -- допплеровская орбитография и радиопозиционирование, интегрированные со спутником
    \item РЛС -- радиолокационная станция
    \item DSST (Draper Semi-analytical Satellite Theory) -- численно-аналитическая баллистическая модель
    \item SSA (Space Situational Awareness) -- система контроля околоземного пространства Европейского космического агентства
    \item НОО -- низкая околоземная орбита
    \item ГСО -- геостационарная орбита
    \item NORAD (North American Aerospace Defense Command) -- Североамериканское командование воздушно-космической обороны
    \item SGP (Simplified General Perturbations) -- упрощенная модель возмущений, используемая NORAD
    \item TLE (Two-Line Elements) -- двухстрочные элементы орбиты, используемые в SGP
    \item EGM2008 (Earth gravitational Model 2008) -- модель гравитационного потенциала Земли
    \item УФ излучение -- ультрафиолетовое излучение
    \item SOLAARS CF (Solar and Atmospheric Adaptive Refraction and Scattering–Controlled Fusion
    -- Солнечное и атмосферное адаптивное преломление и рассеяние) -- динамическая модель тени
    \item TSI (Total Solar Irradiance) -- энергетический поток на расстоянии одной астрономической единицы
    \item МНК -- метод наименьших квадратов
\end{itemize}

\newpage