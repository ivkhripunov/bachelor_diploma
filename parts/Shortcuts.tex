\paragraph{Список обозначений и сокращений}

\begin{itemize}
    \item JD (julian date) -- юлианская дата
    \item MJD (modified julian date) -- модернизированная юлианская дата
    \item TAI (International Atomic Time) -- международное атомное время
    \item UT1 (Universal Time) -- всемирное время
    \item UTC (Universal Time Coordinated) -- координированное всемирное время
    \item TT (Terrestrial Time) -- земное время
    \item TCG (Geocentric Coordinate Time) -- геоцентрическое координатное время
    \item TCB (Barycentric Coordinate Time) -- барицентрическое координатное время
    \item TDB (Barycentric Dynamical Time) -- барицентрическое динамическое время
    \item ICRS (International celestial reference system) -- международная небесная
    система координат
    \item ICRF (International celestial reference
    frame) -- практическая реализация ICRS
    \item GCRS (Geocentric celestial reference system) -- геоцентрическая небесная
    система координат
    \item GCRF (— Geocentric Celestial Reference
    Frame) -- практическая реализацтя GCRF
    \item ITRS (International terrestrial reference system) -- международная земная система координат
    \item ITRF (International Terrestrial Reference
    Frame) -- практическая реализация ITRS
    \item IRP (IERS Reference Pole) -- опорный референсный полюс
    \item CIP (Celestial Intermediate Pole ) -- осредненная ось вращения Земли
    \item LOD (excess length of day) -- избыточная длина суток
    \item КО -- космический объект
    \item КА -- космический аппарат
    \item СК -- система координат
    \item MEE (Modified Equinotical Elements) -- модифицированные равноденственные элементы
    \item ОДУ -- обыкновенное дифференциальное уравнение
    \item DSST (Draper Semi-analytical Satellite Theory) -- численно-аналитическая баллистическая модель
    \item SSA (Space Situational Awareness) -- система контроля околоземного пространства Европейского космического агентства
    \item НОО -- низкая околоземная орбита
    \item ГСО -- геостационарная орбита
    \item NORAD (North American Aerospace Defense Command) -- Североамериканское командование воздушно-космической обороны
    \item SGP (Simplified General Perturbations) -- упрощенная модель возмущений, используемая NORAD
    \item TLE (Two-Line Elements) -- двухстрочные элементы орбиты, используемые в SGP
    \item EGM2008 (Earth gravitational Model 2008) -- модель гравитационного потенциала Земли
    \item УФ излучение -- ультрафиолетовое излучение
    \item SOLAARS CF (Solar and Atmospheric Adaptive Refraction and Scattering Curve Fitting
    -- Солнечное и атмосферное адаптивное преломление и рассеяние) -- динамическая модель тени с интерполяцией
    \item TSI (Total Solar Irradiance) -- энергетический поток на расстоянии одной астрономической единицы
    \item ГНСС -- глобальная навигационная спутниковая система
    \item SLR (Satellite Laser Ranging) -- спутниковая лазерная дальнометрия
    \item DORIS (Doppler Orbitography and Radiopositioning Integrated by Satellite) -- допплеровская орбитография и радиопозиционирование, интегрированные со спутником
    \item РЛС -- радиолокационная станция
    \item МНК -- метод наименьших квадратов
\end{itemize}

\newpage