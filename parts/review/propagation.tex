\subsection{Элементы орбиты}
Вектором состояния X назовем упорядоченную совокупность переменных, полностью определяющих состояние системы в заданный момент времени.
В простейшем случае такой набор состоит из положения $\vec{r}$ и скорости $\vec{v}$ материальной точки. 
Также в этот набор могут входить площадь поверхности и другие параметры космического объекта, оказывающие влияние на его движение.

Однако, в ходе орбитального движения $\vec{r}$ и $\vec{v}$ меняются за виток значительно, что приводит к снижению точности при численном интегрировании.
Поэтому зачастую в ходе решения задачи двух тел в небесной механике используют не радиус-вектор и скорость, а элементы орбиты.
Самыми распространенными из них являются кеплеровы элементы и модифицированные равнодественные элементы (MEE).
В элементах орбиты быстро меняющаяся переменная, описывающая положение КО на орбите, отделена от медленно меняющихся переменных,
определяющих ориентацию и форму орбитальной плоскости.
\begin{center}
    \begin{minipage}[t]{0.45\textwidth}
        \vspace{0pt}
        \textbf{Кеплеровы элементы:}
        \begin{itemize}
            \item наклонение $i$
            \item долгота восходящего узла $\Omega$
            \item аргумент перицентра $\omega$
            \item эксцентриситет $e$
            \item большая полуось $a$
            \item истинная аномалия $\nu$
        \end{itemize}
    \end{minipage}
    \hspace{1cm}
    \begin{minipage}[t]{0.45\textwidth}
        \vspace{0pt}
        \textbf{MEE:}
        \begin{itemize}
            \item $a = a$
            \item $h = e \sin\left(\omega + I \Omega \right)$
            \item $k = e \cos\left(\omega + I \Omega \right)$
            \item $p = \left[\tan{\frac{i}{2}}\right]^I \sin(\Omega)$
            \item $q = \left[\tan{\frac{i}{2}}\right]^I \cos(\Omega)$
            \item $\lambda = M + \omega + I \Omega$
        \end{itemize}
    \end{minipage}
\end{center}

\begin{figure}[h!]
    \centering
    \includegraphics[width=0.6\linewidth]{../images/review/kepler.jpg}
    \captionof{figure}{Кеплеровы элементы орбиты. Центр декартовых координат привязан к центру масс Земли. Ось $Ox$ направлена в точку весеннего равноденствия, 
    ось $Oz$ -- нормаль к плоскости эклиптики, 
    ось $Oy$ дополняет до правой тройки.
    $\vec{N}$ лежит на линии пересечения плоскости эклиптики с плоскостью орбиты.
    $\vec{L}$ -- момент импульса КО,направлен по нормали к орбитальной плоскости. 
    $\vec{e}$ равен по модулю эксцентриситету и направлен на перицентр.}
    \label{fig:kepler}
\end{figure}

Кеплеровы элементы удобны для визуальной интерпретации орбиты (рис. \ref{fig:kepler}).
Первые 3 переменные задают ориентацию орбитальной плоскости в инерциальной системе координат,
эксцентриситет и большая полуось фиксируют форму и размеры эллипса, а истинная аномалия определяет положение КО на орбите.
В качестве последней переменной также могут использоваться эксцентрическая аномалия $E$ и средняя аномалия $M$.
Удобство использования средней аномалии заключается в том, что она меняется со временем равномерно.
Недостаток кеплеровых элементов -- вырожденность при $i = 0$, $i = \pi$ и $e = 0$.
Как следствие, они плохо подходят для интегрирования.

Чтобы избавиться от вырожденности вводится другой набор элементов -- модифицированные равноденственные элементы.
В MEE величина I может принимать два значения:
\[
I = \left\{
\begin{array}{ll}
+1, & \text{если } i < \pi / 2, \\
-1, & \text{если } i \ge \pi / 2
\end{array}
\right.
\]

Также в MEE применяется эксцентрическая долгота $F$ и истинная долгота $L$. 
Они выражаются через кеплеровы элементы следующим образом:
\begin{align*}
    F &= E + \omega + I \Omega \\
    L &= \nu + \omega + I \Omega
\end{align*}

\subsection{Прогноз траектории космического объекта}
Задача прогнозирования движения -- по начальному вектору состояния $X_0$ определить траекторию $X(t)$ объекта.
В основе описания динамики космических аппаратов лежит 2 закон Ньютона, 
поэтому расчет траектории сводится к решению задачи Коши для ОДУ вида:
\[
\begin{cases}
    \dot{X} = f(X, t), \\
    X(t = t_0) = X_0
\end{cases}
\]

При расчете траектории применяются несколько существенно разных подходов. 
Первый из них, аналитический, использует основные факторы, определяющие эволюцию орбиты.
Характерной особенностью аналитических вычислений является низкая ресурсоемкость и невысокая точность.
Таким образом, аналитика обладает высокой качественной предсказательной способностью на коротких временных интервалах, 
«схватывая» главные тренды изменения орбиты.

Численные методы, напротив, позволяют учесть произвольное число сложных возмущающих факторов.
Однако прецизионный численный расчет требует значительно больше вычислений. 
Это связано с ресурсоемкостью расчета правой части ОДУ и,
соответственно, с выбором шага интегрирования для обеспечения заданной точности.

Компромиссом являются полуаналитические подходы, 
в которых используется комбинация численных и аналитических расчетов.
Полуаналитические модели учитывают широкий спектр возмущающих воздействий, 
что позволяет эффективно производить вычисления без потери точности.

Далее приведен краткий обзор основных подходов к прогнозу траектории.

\subsubsection{Аналитический прогноз}
Рассмотрим возмущенную задачу двух тел:

\begin{equation}
    \ddot{\vec{r}} = - \frac{\mu \vec{r}}{r^3} + \vec{f},
    \label{eq:analyt_rv}
\end{equation}
где $\mu$ -- гравитационный параметр Земли, $\vec{f}$ -- возмущающее ускорение, которое может быть разложено по орбитальной СК на радиальную, тангенциальную и нормальную компоненты:

\begin{equation*}
    \vec{f} = R \vec{e_r} + T \vec{e_t} + N \vec{e_n},
\end{equation*}
\begin{align*}
    \vec{e_r} &= \vec{r} / |r| \\
    \vec{e_n} &= \vec{r} \times \vec{v} / |\vec{r} \times \vec{v}| \\
    \vec{e_t} &= \vec{e_n} \times \vec{e_r}
\end{align*}

\begin{figure}[h!]
    \centering
    \includegraphics[width=0.4\linewidth]{../images/review/orbital_system.jpg}
    \captionof{figure}{Орбитальная система}
    \label{fig:orbital_system}
\end{figure}

% Byron, p. 485-486
Преобразуем систему ОДУ (\ref{eq:analyt_rv}) для перехода к кеплеровым элементам.
% \begin{align*}
%     & \frac{da}{dt} = \frac{2 a e^2}{h} \sin(\nu) R 
%                     + \frac{2 a^2 h}{\mu r} T \\
%     & \frac{de}{dt} = \frac{h}{\mu} \left[ \sin(\nu) R +
%                      (e + 2 \cos(\nu) + e \cos^2(\nu)) / (1 + e cos \nu) T \right] \\
%     & \frac{di}{dt} = \frac{r}{h} \cos(\omega + \nu) N \\
%     & \frac{d \Omega}{dt} = \frac{r \sin(\omega + \nu) N}{h \sin(i)} \\
%     & \frac{d \omega}{dt} = - \frac{h}{\mu e} \cos(\nu) R 
%                             - \frac{r}{h} \ctg(i) \sin(\omega + \nu) n
%                             + \frac{(h^2 + r \mu) \sin(\nu)}{\mu e h} T \\
%     & \frac{dM}{dt} = n - \frac{1}{na}\left(\frac{2r}{a} - \frac{1 - e^2}{e} \cos(\nu)\right) R
%                         - \frac{1 - e^2}{nae} \left(1 + \frac{r}{p}\right) sin(\nu) T
% \end{align*}
Если возмущающая сила является потенциальной: $\vec{f} = \nabla R$, то система примет вид:

\begin{align*}
    & \frac{da}{dt} = \frac{2}{na} \frac{\partial R}{\partial M} \\
    & \frac{de}{dt} = \frac{(1 - e^2)^{1 / 2}}{n a^2 e^2}
                    \left((1 - e^2)^{1 / 2} \frac{\partial R}{\partial M} 
                    - \frac{\partial R}{\partial \omega} \right) \\
    & \frac{di}{dt} = \frac{1}{h \sin(i)} \left(cos(i) \frac{\partial R}{\partial \omega} 
                                                        - \frac{\partial R}{\partial \omega} \right) \\
    & \frac{d \Omega}{dt} = \frac{1}{h \sin(i)} \frac{\partial R}{\partial i} \\
    & \frac{d \omega}{dt} = - \frac{\cos(i)}{h \sin(i)} \frac{\partial R}{\partial i}
                            + \frac{(1 - e^2)^{1 / 2}}{n a^2 e^2} \frac{\partial R}{\partial e} \\
    & \frac{dM}{dt} = n - \frac{1 - e^2}{n a^2 e} \frac{\partial R}{\partial e} 
                        - \frac{2}{na} \frac{\partial R}{\partial a}
\end{align*}
где $n = \sqrt{\frac{\mu}{a^3}}$ -- среднее движение, $h = n a^2 (1 - e^2)^2$.

Для построения аналитического решения воспользуемся возмущающим потенциалом от второй гармоники:
 \begin{equation}
    R = - \frac{\mu J_2}{r} \left( \frac{R_\oplus}{r} \right)^2 
        \frac{3}{2} \left(\sin^2(\phi) - \frac{1}{3}\right),
 \end{equation}
где $\phi$ -- широта точки.

Подставив соотношение $sin(\phi) = sin(i) sin(\omega + \nu)$, получим, что $R$ может быть представлена в виде суммы:

\begin{align*}
    & R = R_s + R_p \\
    & R_s = - \frac{3 \mu J_2}{2 r} \left( \frac{R_\oplus}{r} \right)^2 
            \left( \frac{\sin^2(i)}{2} - \frac{1}{3} \right) \\
    & R_p = \frac{3 \mu J_2}{2 r} \left( \frac{R_\oplus}{r} \right)^2
            \frac{\sin^2(i) \cos(2(\omega + \nu))}{2}
\end{align*}

Видно, что первое слагаемое потенциала вызывает постоянное или так называемое вековое возмущение орбиты. 
Период таких возмущений значительно превышает орбитальный период. 
Короткопериодические возмущения, порождаемые слагаемым $R_p$, не приводят к изменениям орбиты на значительном промежутке времени.

Усреднив $R_s$ по периоду, получим:
\begin{equation*}
    R_{avg} =  - \frac{\mu J_2}{2 a} \left( \frac{R_\oplus}{r} \right)^2 
                \left( \frac{3}{4} \sin^2(i) - \frac{1}{2} \right)
                \left( \frac{1}{(1 - e^2)^{3/2}} \right)
\end{equation*}

Подстановка $R_{avg}$ в ОДУ дает вековые возмущения кеплеровых элементов орбиты

\begin{align*}
    & \dot{a}_{sec} = 0 \\
    & \dot{e}_{sec} = 0 \\
    & \dot{i}_{sec} = 0 \\
    & \dot{\Omega}_{sec} = - \frac{3 n R^{2}_\oplus J_2}{2 p^2} \cos(i) \\
    & \dot{\omega}_{sec} = \frac{3 n R^{2}_\oplus J_2}{4 p^2} (4 - 5 \sin^2(i)) \\
    & \dot{M_0}_{sec} = - \frac{3 n R^{2}_\oplus J_2 \sqrt{1 - e^2}}{4 p ^2}
                            (3 \sin^2(i) - 2)
\end{align*}

Аналогичным образом могут быть выделены короткопериодические возмущения. В частности:

\begin{equation*}
    \delta a = \gamma_3 a \left[ (3z \sin^2(\omega + \nu) - 1) \left(\frac{a}{r}\right)^3
                                    - \frac{3z - 2}{2 \eta^3}  \right],
\end{equation*}
где $\gamma_3 = -J_2 \left(\frac{R_\oplus}{a}\right)^2$, $\eta = \sqrt{1 - e^2}$, $z = \sin^2(i)$.

Так как большая полуось, эксцентриситет и наклонение не испытывают вековых возмущений, 
долгота восходящего узла и аргумент перицентра легко интегрируются аналитически.
\begin{align*}
    & \Omega(t) = \Omega_0 + \Omega_{sec} (t - t_0) \\
    & \omega(t) = \omega_0 + \omega_{sec} (t - t_0) \\
\end{align*}
Для получения выражения для $a$ необходимо провести процедуру усреднения среднего движения
\begin{equation*}
    a = \bar{a} + \delta a \rightarrow  \bar{a} = a_0 - \delta a_0
\end{equation*}
\begin{equation*}
    \bar{n} = \sqrt{\frac{\mu}{\bar{a}^3}}
\end{equation*}

В результате получим:
\begin{equation*}
    M(t) = M_0 + (\bar{n} + \dot{M_0}_{sec}) (t - t_0)
\end{equation*}

Более детальное решение задачи аналитического расчета траектории представлено в серии моделей SGP.
Модели используют данные в формате TLE, предоставляемые американской службой NORAD. 
В TLE содержатся не только средние кеплеровы элементы орбиты, но и первая и вторая производные среднего движения.
Модели движения SGP аналитически учитывают возмущения от сжатия Земли, сопротивления атмосферы, гравитации Луны и Солнца.
Из-за сильного влияния атмосферного торможения ошибка прогноза на низких орбитах составляет порядка 1 километра в день.
Для средних и высоких орбит ошибка значительно меньше -- несколько сотен метров на недельном интервале. 

\subsubsection{Численно-аналитический прогноз}
Стимулом к развитию численно-аналитических методов послужил быстрый рост количества
объектов в околоземном пространстве и необходимость их непрерывного отслеживания и каталогизации.
Для таких задач аналитические методы не удовлетворяют требуемой точности, а численные методы
не подходят в силу высокой ресурсоемкости. Численно-аналитические методы, в свою очередь,
объединяют точность и быстродействие за счет гибкой настройки модели движения.

В основе численно-аналитических моделей лежит разделение возмущений на вековые и короткопериодические.
На начальном этапе происходит усреднение орбитальных элементов, 
чтобы исключить высокочастотные возмущения орбиты. Эта операция позволяет в дальнейшем
интегрировать медленно меняющиеся средние элементы с большим шагом (порядка половины дня).
На заключительном этапе прогноза мгновенные значения элементов орбиты вычисляются аналитически по средним элементам.

Частным случаем численно аналитических методов является модель DSST, разработанная для
системы контроля околоземного пространства Европейского космического агентства (SSA).
Математическая модель DSST опирается на методы усреднения и вариации параметров.

Усредненные уравнения движения для консервативной возмущающей силы:

\begin{equation*}
    \frac{d\bar{c}_i}{dt} = -\sum_{j=1}^{6} \left\{\bar{c}_i, \bar{c}_j \right\} 
                                \frac{\partial \bar{R}}{\partial\bar{c}_j} \qquad i=1 \dots 5
\end{equation*}

При наличии неконсервативной силы правая часть дополнительно усредняется по витку:

\begin{equation*}
    \frac{d\bar{c}_i}{dt} = \frac{1}{2 \pi} \int_{0}^{2 \pi} \frac{\partial\bar{c}_i}{\partial\dot{\vec{r}}} 
                        \cdot \vec{Q} d\lambda \qquad i=1 \dots 5
\end{equation*}

Выражение для быстроменяющейся средней долготы:

\begin{equation*}
    \frac{d \bar{\lambda}}{dt} = \frac{d\bar{c}_6}{dt} = 
        \bar{n} -\sum_{j=1}^{6} \left\{\bar{c}_6, \bar{c}_j \right\} 
                                \frac{\partial \bar{R}}{\partial\bar{c}_j}
                +\frac{1}{2 \pi} \int_{0}^{2 \pi} \frac{\partial\bar{c}_6}{\partial\dot{\vec{r}}} 
                        \cdot \vec{Q} d\lambda
\end{equation*}

Переход к мгновенным элементам:

\begin{equation*}
    c_i = \bar{c}_i + \sum_{j=1}^{N} e^j \eta_{i, j}\left(\bar{a}, \bar{\lambda}\right) \qquad i = 1 \dots 6
\end{equation*}

В последних уравнениях были введены следующие обозначения:
\begin{align*}
    \bar{c}_{i=1 \dots 6} &: \text{средние равноденственные элементы} \left[\bar{h}, \bar{k}, \bar{k}, \bar{p}, \bar{q}, \bar{\lambda}\right] \\
    \bar{R} &: \text{усредненный возмущающий потенциал для консервативной силы} \\
    \vec{Q} &: \text{неконсервативная сила} \\
    \bar{n} &: \text{усредненное среднее движение} \\
    \left\{\bar{c}_i, \bar{c}_j \right\} &: \text{скобки Пуассона} \\
    \eta_{i, j} &: 2\pi \text{-периодические функции}
\end{align*}

Точность прогноза по модели DSST отличается для разных классов орбит.
Среднеквадратичное отклонение при сравнении с численным расчетом на 7 суток составляет от 10 метров для НОО до 20 метров для ГСО.
Для высокоэллиптических орбит среднеквадратичная ошибка может достигать 75 метров.

\subsubsection{Численный прогноз}

\paragraph{Модель вращения Земли}

\paragraph{Геопотенциал}

\paragraph{Сопротивление атмосферы}

\paragraph{Солнечное давление}

\paragraph{Альбедо}