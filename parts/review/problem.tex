\subsection{Проблематика}
На сегодняшний день на околоземных орбитах находится большое число космических объектов.
Среди них есть не только активные аппараты, но и космический мусор, образовавшийся
в результате фрагментации корпусов ракет и спутников.
Растущая загруженность околоземной среды требует развития эффективных средств контроля 
космического пространства, в частности, для уточнения орбит КО.
В основе восстановления орбиты лежит задача прогнозирования движения.
Для построения траектории необходимо вычислять силы, действующие на космический объект.

\subsubsection{Оценка быстродействия}
Краткий обзор вычислительных сложностей приведен в таблице \ref{tab:force_perf}.
Сила гравитационного притяжения вычислялась с использованием модели EGM2008 для
разного числа гармоник. За единицу принята трудоемкость подсчета с учетом 16*16 гармоник.

\begin{table}[h!]
\caption{Сравнение времени вычисления некоторых сил, действующих на КО}
\renewcommand{\arraystretch}{1.5}
\begin{tabular}{|cc|c|l}
\cline{1-3}
\multicolumn{2}{|c|}{Сила}                & Время расчёта, усл. ед. &  \\ \cline{1-3}
\multicolumn{1}{|c|}{\multirow{3}{*}{Гравитационное притяжения}} &
  16*16 гармоник &
  1 &
   \\ \cline{2-3}
\multicolumn{1}{|c|}{} & 64*64 гармоник   & 12.3                    &  \\ \cline{2-3}
\multicolumn{1}{|c|}{} &
  \begin{tabular}[c]{@{}c@{}}64*64 гармоник (ускорение)\end{tabular} &
  1.3 &
   \\ \cline{1-3}
\multicolumn{1}{|c|}{\multirow{3}{*}{Сопротивление атмосферы}} &
  ГОСТ Р 25645.166-2004 &
  2.2 &
   \\ \cline{2-3}
\multicolumn{1}{|c|}{} & NRLMSISE-00      & 14                      &  \\ \cline{2-3}
\multicolumn{1}{|c|}{} & NRLMSIS 2.1      & 86                      &  \\ \cline{1-3}
\multicolumn{1}{|c|}{\multirow{2}{*}{Солнечное давление}} &
  точечного источника &
  0.06 &
   \\ \cline{2-3}
\multicolumn{1}{|c|}{} & непрерывной тени & 0.1                     &  \\ \cline{1-3}
\end{tabular}%
\label{tab:force_perf}
\end{table}

Среди рассмотренных сил на низкой орбите основной вклад вносят гравитационное притяжение Земли 
и сопротивление атмосферы.
В классических подходах наиболее ресурсоемким этапом является вычисление силы гравитационного 
притяжения Земли. Ресурсоемкость вычисления данной силы с ростом числа гармоник растет квадратичною.
Соответствующие затраты для 64*64 гармоник составили 12 условных единиц.
Однако использование интерполяционного подхода \cite{kuznetsov2023} позволяет существенно 
снизить ресурсоемкость расчета гравитационной силы. С учетом интерполяции время
расчета силы притяжения равно времени, затрачиваемому на прямой расчет гравитационной силы
с учетом 16 гармоник потенциала. 

Таким образом, именно расчет силы 
сопротивления атмосферы становится ключевым с точки зрения ресурсоемкости программы.
Вычисление аэродинамического сопротивления атмосферы связано 
с определением ее плотности.
Для повышения эффективности программного комплекса требуется ускорение расчета атмосферной плотности.
Алгоритм, позволяющий осуществить это, рассмотрен в следующем разделе.

\subsubsection{Высокоточное восстановление орбиты}
Точность восстановления орбиты КО зависит от динамической модели и измерений.
Для систем, использующих измерения высокой точности, например, лазерную дальнометрию,
качества прямого моделирования сил начинает не хватать на интервалах по времени
больше дня. Это связано с тем, что моделирование
любой силы, действующей на КО, содержит определенную ошибку. Особенно сильно эта ошибка
проявляется при расчете давления солнечного и теплового излучения и эффекта альбедо. 
Величина этих сил определяется ориентацией КО. Если объект обладает сложной конструктивной геометрией,
то вычислить площадь поверхности, на которую падает излучение, и определение модуля силы затруднено. 

Для того, чтобы преодолеть это, вводится дополнительная параметризация модели движения,
реализация которой будет представлена далее.