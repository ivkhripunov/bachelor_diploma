\subsection{Типы измерений}

Для уточнения орбиты требуются измерения параметров, связанных с положением и движением КО.
Конкретный набор измеряемых параметров зависит от типа используемых наблюдательных средств.
Рассмотрим несколько классических наборов измеряемых величин.

\subsubsection{Односторонние измерения дальности}
Расстояние между спутником и станцией наблюдения является одним из наиболее часто измеряемых параметров.
Такая модель измерений получила широкое распространение во многом благодаря глобальным навигационным спутниковым системам.
Каждый спутник ГНСС оснащен антенной, которая излучает электромагнитные волны на нескольких частотах.
Сигнал каждого спутника модулируется особым образом, чтобы приемник мог определить момент времени $T_{T}$ излучения волны по шкале времени спутника
и соответствующие этому моменту координаты излучателя. Момент приема сигнала $T_{R}$ фиксируется по часам приемника.

Зная разницу между временем отправки и приема сигнала, а также используя свойство прямолинейности распространения света, можно 
рассчитать величину, называемую псевдодальностью:
\begin{equation*}
    \rho = (T_{R} - T_{T}) c,
\end{equation*}
где $c$ -- скорость света.

Псевдодальность не совпадает с геометрической в силу несогласованности часов излучателя и приемника, 
особенностей распространения сигнала в атмосфере и относительного движения излучателя и приемника.

Учет этих факторов необходим при формировании расчетного аналога измерения:

\begin{equation*}
    \tilde{\rho} = |\vec{r}_{T} - \vec{r}_{R}| 
                    + c \left( \delta t_{T} - \delta t_{R} \right)
                    + \delta \rho_{tropo} + \delta \rho_{ion} + \epsilon,
\end{equation*}
где помимо геометрической дальности $|\vec{r}_{T} - \vec{r}_{R}|$ присутствуют слагаемые,
связанные с поправкой часов спутника $\delta t_{T}$ и станции $\delta t_{R}$, тропосферная $\delta \rho_{tropo}$ и ионосферная $\delta \rho_{ion}$ задержки.
Благодаря измерениям на нескольких частотах ионосферная задержка может быть с высокой точностью исключена.
В $\epsilon$ включаются остаточные ошибки, связанные с неучитываемыми нелинейными эффектами.

Для повышения точности позиционирования и уменьшения разброса результатов
в ходе решения навигационной задачи могут обрабатываться не только псевдодальности,
но и фазовые измерения сигнала. 
В этом случае точность позиционирования с использованием ГНСС может достигать 10 сантиметров.

\subsubsection{Двусторонние измерения дальности}
В ходе двустороннего измерения дальности фиксируется время, за которое излученный сигнал
достигает цели, отражается и возвращается в точку испускания. 
Вариантом такой измерительной системы является лазерная дальнометрия (SLR).
В отличии от односторонних измерений дальности в лазерной дальнометрии излучатель и приемник
находятся в одном месте и подключены к одним часам, что избавляет от необходимости уточнения поправок шкал времени.
При этом атмосферные поправки все еще требуются. 
В качестве измерений усредняется расстояние, пройденное сигналом в прямом и обратном направлениях:

\begin{equation*}
    \rho_{avg} = \frac{1}{2} \left[ 
        \left(T_{R} - T_{T}\right) c + \delta \rho_{atm} + \epsilon \right]
\end{equation*}

Современные станции SLR используют лазеры с длиной волны 532 нм, что соответствует оптическому диапазону.
С этим связан недостаток лазерной дальнометрии -- зависимость от погодных условий.

В качестве примера применения SLR рассмотрим серию аппаратов LAGEOS (LAser GEOdynamics Satellite).
Аппараты LAGEOS-I и LAGEOS-II были запущены на среднюю околоземную орбиту в 1976 и 1992 годах соответственно.
Цель миссии -- изучение геодинамики, в частности, определение формы земной поверхности и уточнение параметров вращения Земли. 
Каждый аппарат имеет шарообразную форму и оснащен набором уголковых отражателей, необходимых
для точного отражения лазерного сигнала.

Позиционирование спутников выполняется на основе измерений наблюдательных пунктов Международной службы лазерной дальнометрии.
Ошибки лазерных измерений составляют менее 1 сантиметра, что позволяет восстанавливать орбиту аппаратов с точностью до нескольких сантиметров.

\subsubsection{Оптические измерения}


\subsubsection{Лазерные}