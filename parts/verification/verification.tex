\section{Верификация}
\label{sec:Chapter3} \index{Chapter2}

Проведено тестирование точности различных конфигураций интерполянтов для модели NRLMSISE-00.

Диапазон интерполяции по расстоянию был выбран от 350 до 800 километров, интервал по времени
составил одни сутки. В ходе тестов была использована космическая погода за 11 января 2014 года.
Этот год соответствует высокой солнечной активности, что позволяет протестировать качество
интерполяции в сложных условиях.

Сетка для тестирования алгоритма состояла из более 1 миллиона точек, равномерно
распределенных по области интерполяции. Для каждого интерполянта измерялась максимальная
ошибка на тестовой сетке, размер и величина ускорения по сравнению с прямым расчетом плотности.

Полный перебор по возможным конфигурациям и количеству ячеек интерполянтов сложен
из-за большого количества вариантов, времени, затрачиваемого на построение каждого интерполянта и
занимаемой памяти. Поэтому поиск подходящих параметров проводился поэтапно.

Учитывая близкую к линейной зависимость логарифма плотности от высоты, для этой координаты
была выбрана наименьшая степень полинома -- 2. По времени была установлена степень 7,
чтобы добиться высокой точности при меньшем количестве точек и иметь возможность
построения интерполянта на более длительные интервалы времени без потери точности. 
В рассмотрении остались наборы (2, 3, 5, 7) и (2, 5, 3, 7). Интерполянты для полиномов более
высоких порядков не строились из-за снижения производительности таких конфигураций и
существенного повышения затрат памяти для хранения коэффициентов.

Для каждого набора было зафиксировано количество ячеек по высоте -- 65. Количество ячеек
по широте, долготе и времени варьировалось от 5 до 50. Эмпирически определялось
количество ячеек для достижения точности при заданном разбиении по высоте. Далее
для интерполянтов с параметрами, выбранными на предыдущем шаге, варьировалось
количество ячеек по высоте и определялась точность, достижимая с такими параметрами.


\newpage