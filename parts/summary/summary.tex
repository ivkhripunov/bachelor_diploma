\section{Выводы}
\label{sec:Chapter5} \index{Chapter2}
В настоящей работе автор предложил способ ускорения расчета плотности атмосферы Земли с 
использованием четырехмерной интерполяции. Разработанный алгоритм позволяет
быстро и точно прогнозировать траектории объектов в околоземном пространстве. 
Кроме того, автором была реализована параметризация модели движения для прецизионного
восстановления орбиты. Проведена валидация разработанного программного комплекса на 
орбите космического аппарата LAGEOS-2.

В качестве дальнейшего направления работы можно выделить совершенствование методики
поиска рациональных параметров интерполянта, обеспечивающих приемлемую точность прогноза
при минимальных затратах памяти.
\newpage