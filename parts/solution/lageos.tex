\section{Параметризация модели движения}
\label{sec:Chapter2} \index{Chapter2}

Параметризованная модель движения представима в следующем виде:

\begin{equation*}
    \ddot{\mathbf{x}} = \mathbf{f}(\mathbf{x}, t) +
    \mathbf{f}_1(a_1, \dots, a_n, t),
\end{equation*}
где к исходной системе уравнений орбитальной динамики добавлена
некоторая функция, зависящая от $n$ параметров.

Неизвестные величины $a_1, \dots, a_n$ должны определяться в ходе процедуры минимизации при
восстановлении орбиты.

В настоящей работе исследуется две параметризации -- псевдоускорения и псевдоимпульсы.

\subsection{Псевдоускорения}

Псевдоускорение -- это дополнительная сила которая действует на КО на некотором участке 
траектории. Как правило, мерный интервал разбивается на несколько участков, и на каждом
из них дополнительная сила постоянна. В некоторых случаях ускорения выбрают не
кусочно-постоянными, а кусочно-линейными для непрерывности производной силы.
Направление ускорений может быть разложено по компонентам орбитальной системы координат на радиальную, тангенциальную и нормальную
составляющие. 

Концептуальная схема применения псевдоускорений изображена на рис. !!!.

Уравнение движение с учетом псевдоимпульсов:
\begin{equation*}
    \ddot{\mathbf{x}} = \mathbf{f}(\mathbf{x}, t) +
        \sum_{i=0}^{n-1} \sum_{j=1}^{3} a_{i,j} \cdot \xi_{i}(t) \cdot \mathbf{e_j} (t),
\end{equation*}
где $a_{i,j}$ -- величина импульса вдоль направления $\mathbf{e_j}$.

\begin{equation*}
    \xi_{i}(t) = 
            \begin{cases}
                0, & t < t_i \\
                1, & t_i < t < t_{i+1} \\
                0, & t_{i+1} \le t
            \end{cases}
\end{equation*}

Столбец матрицы изохронных производных, соответствующий параметру псевдоимпульса,
может быть вычислен в процессе интегрирования уравнения в вариациях аналогично
столбцу производных по площади аэродинамического сопротивления или солнечного давления.

\subsection{Псевдоимпульсы}

Псевдоимпульсы -- мгновенные изменения скорости в определенные моменты времени.
Эти изменения выражаются в разрыве I рода скорости, что приводит к появлению
$\delta$-функции в ускорении. При этом траектория остается непрерывной.

Схематичный вид псевдоимпульсов приведен на рис. !!!.

Динамика движения при наличии псевдоимпульсов описывается выражением:
\begin{equation*}
    \ddot{\mathbf{x}} = \mathbf{f}(\mathbf{x}, t) +
        \sum_{i=0}^{n-1} \sum_{j=1}^{3} v_{i,j} 
                    \cdot \delta (t - t_i) \cdot \mathbf{e_j} (t_i),
\end{equation*}
где $v_{i,j}$ -- величина импульса вдоль направления $\mathbf{e_j}$.

Получим выражение для столбца в матрице изохронных производных, соответствующего одному импульсу.
Введем функцию $U(t)$ как блок размерами 6 на 6 матрицы изохронных производных $\Phi$, отвечающий за
матрицу Якоби мгновенных элементов орбиты $\mathbf{\textbf{Э}}(t)$ по начальным элементам 
$\mathbf{\textbf{Э}}_0$:

\begin{equation*}
    U(t) = \frac{\partial \mathbf{\textbf{Э}}(t)}{\partial \mathbf{\textbf{Э}}_0}
\end{equation*}

Пусть в момент времени $t_0$ был совершен псевдоимпульс.
Представим матрицу Якоби мгновенных элементов орбиты по элементам в момент импульса через $U(t)$:
\begin{equation*}
    \frac{\partial \mathbf{\textbf{Э}}(t)}{\partial \mathbf{\textbf{Э}}(t_0)} = 
    \frac{\partial \mathbf{\textbf{Э}}(t)}{\partial \mathbf{\textbf{Э}}_0} \cdot
    \frac{\partial \mathbf{\textbf{Э}}_0}{\partial \mathbf{\textbf{Э}}(t_0)} = 
    U(t) \cdot U(t_0)^{-1}
\end{equation*}

Используя полученные уравнения, можно получить искомый столбец при $t > t_0$:
\begin{equation*}
    \frac{\partial \mathbf{\textbf{Э}}(t)}{\partial \mathbf{\textbf{Э}}(t_0)} = 
    \frac{\partial \mathbf{\textbf{Э}}(t)}{\partial \mathbf{\textbf{Э}}(t_0)} \cdot
    \frac{\partial \mathbf{\textbf{Э}(t_0)}}{\partial \mathbf{v}_{t_0}} =
    U(t) \cdot U(t_0)^{-1} \cdot
    \frac{\partial \mathbf{\textbf{Э}(t_0)}}{\partial \mathbf{v}_{t_0}}
\end{equation*}

При $t < t_0$ данный столбец равен соответствующему столбцу единичной матрицы, 
так как импульс еще не был соверешен.

Заметим, что рассмотренные столбцы могут быть вычислены с минимальными дополнительными
ресурсами, так как используется уже имеющееся после прогноза решение уравения в вариациях.

\subsection{Верификация}

Тестирование программного комплекса проводилась на КА LAGEOS-2.
Измерения орбиты данного аппарата выполняются с помощью лазерной дальнометрии.
Высокоточные эфемериды LAGEOS-2 публикуются рядом агентств, 
в том числе итальянским и немецким исследовательскими центрами. Эти орбиты
были выбраны в качестве референсных.

Характеристики аппарата и общие сведения об орбите отражены в таблицах \ref{tab:lageos2}
и \ref{tab:lageos2_orb}.

\begin{table}[h!]
    \centering
    \begin{minipage}[t]{0.48\textwidth}
    \centering
    \begin{tabular}{|l|l|}
    \hline
    \multicolumn{2}{|c|}{\textbf{Параметры аппарата}} \\ \hline
    Форма       & Сфера      \\ \hline
    Диаметр     & 60 см      \\ \hline
    Вес         & 405.38 кг  \\ \hline
    \end{tabular}
    \caption{Параметры аппарата LAGEOS-2}
    \label{tab:lageos2}
    \end{minipage}
    \hfill
    \begin{minipage}[t]{0.48\textwidth}
    \centering
    \begin{tabular}{|l|l|}
    \hline
    \multicolumn{2}{|c|}{\textbf{Параметры орбиты}} \\ \hline
    Перигей        & 5620 км   \\ \hline
    Наклонение     & 52.64°    \\ \hline
    Эксцентриситет & 0.0135    \\ \hline
    Период         & 223 мин.  \\ \hline
    \end{tabular}
    \caption{Параметры орбиты LAGEOS-2}
    \label{tab:lageos2_orb}
    \end{minipage}
\end{table}

Для тестирования была реализована модель движения, близкая модели к модели агентства.
Ключевые характеристики модели представлены в таблице !!!.